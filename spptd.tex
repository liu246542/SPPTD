\documentclass[conference]{IEEEtran}
\IEEEoverridecommandlockouts
% The preceding line is only needed to identify funding in the first footnote. If that is unneeded, please comment it out.
\usepackage{cite}
\usepackage{amsmath,amssymb,amsfonts}
\usepackage{algorithmic}
\usepackage{graphicx}
\usepackage{textcomp}
\usepackage[OT1]{fontenc}
\usepackage{xcolor}
% \usepackage[colorlinks]{hyperref}
\def\BibTeX{{\rm B\kern-.05em{\sc i\kern-.025em b}\kern-.08em
    T\kern-.1667em\lower.7ex\hbox{E}\kern-.125emX}}
\begin{document}

\title{Privacy-Preserving Truth Discovery for Sparse Data in Mobile Crowdsensing Systems\\
% \title{Privacy-preserving truth discovery in sparse mobile crowd sensing systems\\
% {\footnotesize \textsuperscript{*}Note: Sub-titles are not captured in Xplore and
% should not be used}
% \thanks{Identify applicable funding agency here. If none, delete this.}
}

\author{\IEEEauthorblockN{xxx}
\IEEEauthorblockA{\textit{Department of Electronic and Engineering} \\
\textit{University of Science and Technology of China}\\
Hefei, Anhui 230027, China \\
xxx@mail.ustc.edu.cn}
}
\maketitle

\begin{abstract}
% 1.Motivation 2.Problem Statement 3.Approach 4.Results 5.Conclusions

\end{abstract}

\begin{IEEEkeywords}
Crowd Sensing, Truth Discovery, Privacy Preservation
\end{IEEEkeywords}

\section{Introduction}~\label{sec1}
% What is the problem, why is the problem is important (motivation)
% 介绍 mobile crowdsensing
Recently, developments in the field of cloud computing and Internet of Things (IoT) have led to a growing interest in mobile crowdsensing (MCS).
In a typical mobile crowdsensing systems, the back-end server (or cloud) can collect and analyze the sensory data from different mobile devices.
 % sensory data are collected by different mobile devices (e.g., smartphones, wearables) with a highly dynamic nature.
% the sensory data are collected from different mobile devices, and the 
% the back-end server (or cloud) 
% Benefiting from crowd sensing, a cloud server can collect and analyze the sensory data from different terminal devices (usually called sensing workers).
However, the quality of sensory data collected by different mobile devices may vary greatly, it is difficult for cloud servers to derive reliable aggregated result (usually referred to \textit{ground truth}).
To address this challenge, a series of methods called truth discovery~\cite{li_resolving_2014,li_confidence-aware_2014} have been proposed, where the general principle is estimating each mobile device's reliability degree (usually referred to weight) before calculating the ground truth.
% 估计权重,在聚合数据之前
% an effective method 
% received more and more attention.

Though truth discovery can help cloud servers to derive ground truth in an effective way, it poses threats to workers' privacy directly.
To achieve privacy preservation, a considerable amount of work has been proposed for privacy-preserving truth discovery (PPTD)~\cite{miao_cloud-enabled_2015,xu_efficient_2019,zhang_reliable_2019,xue_inpptd_2020}.
% 大多数 基于 CRH, 默认 所有终端 观测 所有对象数据,引出我们的 contribution
Most of these approaches are based on CRH algorithm~\cite{li_resolving_2014}, which is a representative algorithm of truth discovery in recent years.
The core principle of CRH is that the mobile device will be assigned a higher weight if the sensory values provided by him/her are closer to the ground truth, and the sensory values provided by a device will be counted more if this device has a higher weight~\cite{miao_cloud-enabled_2015,xu_efficient_2019}.
Since the CRH algorithm requires that each worker has to send values of all sensed objects.
% It seems impractical in real mobile sensing systems because 
This method will face challenge if there is a worker only sends values of partial sensed objects.
Taking the sparse mobile crowd sensing~\cite{wang_sparse_2016} as an example, the sensed objects represent different sub-areas, which infers the sensory values provided by workers are sparse.
So, it is infeasible to apply these CRH-based PPTD algorithms in this situation.

% contribution
In this paper, we propose a privacy-preserving truth discovery to deal with the situation when the sensory data are sparse.
In particular, our contributions are summarized in the following:

\begin{itemize}
  \item We present a novel PPTD framework which address the issues of data sparsity based on the CATD algorithm.
  \item With the help of secure two-party computation, our proposed scheme provide strong privacy for workers in an efficient way.
  \item We conduct extensive experiments to evaluate the performance of our proposed scheme, and the results demonstrate that our design is practical in crowd sensing systems.
\end{itemize}

The remainder of this paper is organized as follows. ......


\section{Related Work}~\label{sec2}
% 基于 CRH
% 基于 CATD
% 为什么它们都不行

At the early stage of the study, the single cloud server setting was widely used in PPTD.
The first serious discussions of privacy preservation for truth discovery was presented by Miao et al.~\cite{miao_cloud-enabled_2015}.
They used threshold homomorphic cryptography to encrypt the sensory data and protect works' privacy. 
But the computation overhead of homomorphic cryptography is huge, Xu et al.~\cite{xu_efficient_2019} proposed an efficient scheme based on double-masking which is a fault-tolerance perturbation-based protocol.
% While 单服务器
When under the single cloud server setting, workers have to participate the iterations, so that workers need to suffer additional computing and communication overhead.

To reduce workers' overhead, Miao et al.~\cite{miao_lightweight_2017} proposed $L$-PPTD and $L^2$-PPTD by involving two non-colluding cloud servers.
% Different from the single cloud server setting, workers 
After that, more and more schemes are designed under two non-colluding cloud servers setting (such as~\cite{zhang_lptd_2019,zhang_reliable_2019,xue_inpptd_2020,tang_achieving_2021}).

On the other hand, most of the existing privacy-preserving truth discovery schemes are based on CRH algorithm, which requires workers to provide sensory data for all objects.
But in the real-world mobile crowdsensing systems (like sparse mobile crowd sensing~\cite{wang_sparse_2016}), it is usually impractical for workers to observe all objects at a specific moment.
The issue of data sparsity in mobile crowdsensing is getting more and more attention from recent years, especially when taking privacy preservation into account.
% According to research~\cite{wang_sparse_2020}, it is important not only to protect the sensory data observed by workers, but also to protect what objects observed by workers.
% There are two types of privacy need to protect, one is the data observed by workers, the other is which objects observed by workers.
Fortunately, there is another truth discovery algorithm called CATD~\cite{li_confidence-aware_2014} for handling long-tail data, which is also suitable for the sparse data situation.
% In comparison, CATD~\cite{li_confidence-aware_2014} is another truth discovery algorithm for handling long-tail data, which is also suitable for the sparse data situation.
Zheng et al.~\cite{zheng_learning_2018} designed an encrypted truth discovery based on CATD under two non-colluding cloud servers where all procedures are conducted in the encrypted domain.
However, according to research~\cite{wang_sparse_2020}, it is important not only to protect the sensory data observed by workers, but also to protect what objects observed by workers.
% As depicted in Figure xxx, 
% And the existing privacy-preserving scheme based on CATD~\cite{zheng_learning_2018} 
% But the existing privacy-preserving scheme based on CATD~\cite{zheng_learning_2018} only considered the privacy of sensory data 
There still lacks research on the two types of privayc for truth discovery in sparse crowd sensing.

\section{System Model and Design Goals}\label{sec3}

In this section, we present the system model and formulate the problem we target in this paper.
% sparse data problem in MCS.
\subsection{System Model}\label{sec3-A}

Generally, our system model consists of two types of entities: a number of workers and two cloud servers. %用图说明
The specific explanations of these entities are listed as follows.
% worker 的作用, cloud server 的作用
\begin{itemize}
  \item \textbf{Workers:} Workers refers to the data providers with mobile devices (e.g., smartphones, wearables) in MCS. They are responsible for collecting sensory data of different objects. After that, these sensory data are uploaded to two cloud servers.  Since most of mobile devices are resource-limited, it is not necessary for each worker to collect sensory data of all objects.
  \item \textbf{Clouds:} Clouds are responsible for analysing the sensory data of different objects. More specifically, after receiving the uploaded sensory data, these two cloud servers start to estimate the ground truth for each sensory object. We assume that the two cloud servers have sufficient computation and storage capabilities.
  % the agency to analyse the 
\end{itemize}
% Workers are responsible for collecting sensory data of different objects.
% After that, workers need to upload their sensory data to two cloud servers.
In this paper, suppose that there are $M$ objects need to be collected, and the number of workers is $K$.
We denote $x_m^k$ as the sensory data of object $m$ of worker $k$.
Moreover, worker $k$ also maintains an indication vector $\Phi_k = \{\phi_1^k, \ldots, \phi_M^k\}$ to mark the missing sensory objects in which $\phi_m^k = 1$ means worker $k$ has collected the sensory data of object $m$, and $\phi_m^k = 0$ otherwise.
Each worker uploads the sensory data and task vector to clouds.
After that, two cloud servers calculate each worker's weight (denoted by $w_k$ for worker $k$), and finally estimate the ground truth for each object (denoted by $x_m^*$ for object $m$). 
% The goal of two cloud servers is to estimate the ground truth $X^* = \{x_1^*, x_2^*, \ldots, x_M^*\}$ for each sensory object, where $x_m^*$ represents the estimated ground truth for $m$-th object.
\subsection{Threat Model and Design Goals}

In our proposed scheme, we assume that all entities are semi-honest, which means that no matter workers or coulds will honestly execute the protocols, but they are also curious about participants' privacy, such as workers' sensory data, worker's weights.
Moreover, the two cloud servers in our scheme are non-colluding, which is a common assumption in most two-server models~\cite{zhang_lptd_2019,zhang_reliable_2019}.
% 不考虑 恶意 workers, 不考虑 lazy workers
Note that the lazy workers are not considered, since this issue can be solved by integrating incentive mechanism~\cite{xue_inpptd_2020}.
% The consideration of malicious workers are also not involved, 
% \section{Problem Statement}
% In this section, we describe the truth discovery and formulate the sparse data problem in crowd sensing systems.

The main purpose of our shceme is to protect workers' privacy.
Since the sensory data providered by workers are sparse, we should not only consider the privacy of sensory data, but also consider the privacy of indication vectors.

% there are two kind of privacy need to be considered.
% The first one is the 
% of workers while the 
% workers' privacy while they 
% protecting workers' privacy 
% The design goals of our proposed scheme can be 

% \subsection{Background on Truth Discovery}

% In real-world crowdsensing applications, truth discovery is 
% Weight Update:

% $$w_k = \frac{\chi_{(1-\alpha/2, \sum_{m=1}^M \phi_m^k)}^2}{\sum_M (x_m^k - x_m^*)^2}$$

% Truth Update:

% $$x_m^* = \frac{\sum_K w_k \phi_m^k x_m^k }{\sum_K \phi_m^k x_m^k}$$

% \subsection{System Model}


\section{Preliminaries}\label{sec4}

In this section, we introduce the truth discovery algorithm CATD~\cite{li_confidence-aware_2014} and additively homomorphic cryptosystem.
% ABY2.0~\cite{patra_aby20_2020}.

% \subsection{Homomorphic Encryption}

\subsection{CATD}

% CATD 强调与 worker 提交的数量,。。。
In general, truth discovery algorithm consists of two parts: weight update and truth update.

\textit{1) Weight Update:} Given the ground truth, the weight for each worker $k$ is computed as
\begin{equation}
w_k = \frac{\chi^2_{(1-\alpha/2,\sum_{m=1}^M \phi_m^k)}}{\sum_{m=1}^M \phi_m^k(x_m^k - x_m^*)^2}
\end{equation}
Note that $\chi^2$ denotes the Chi-squared distribution, and the constant $\alpha$ denotes the significant level which is usually a small number such as 0.05.

\textit{2) Truth Update:} Given the weight $w_k$ for each worker $k$, the truth for each object $m$ is computed as
\begin{equation}
x_m^* = \frac{\sum_{k=1}^K \phi_m^k w_k x_m^k}{\sum_{k=1}^K \phi_m^k w_k }
\end{equation}

In real applications, to achieve more accurate ground truth values, weight update and truth update should be executed iteratively until the convergence criteria is met.

\subsection{Additively Homomorphic Cryptosystem}\label{sec4-b}

We adopt additive homomorphic cryptosystem in our proposed scheme.
Generally, an additive homomorphic cryptosystem consists of the following probabilistic poly-time algorithms.

\begin{itemize}
  \item $\mathsf{Setup}(1^\kappa)\to pp$: Taken the input of security parameter $\kappa$, the algorithm returns the public parameter $pp$.
  \item $\mathsf{KeyGen}(1^\kappa)\to (pk, sk)$: Taken the input of security parameter $\kappa$, the algorithm returns the public key $pk$ and private key $sk$.
  \item $\mathsf{Enc}_{pk}(m)\to c$: Given the message $m$, the encryption algorithm outputs $c$ which is the ciphertext of message $m$.
  \item $\mathsf{Dec}_{sk}(c)\to m$: Given the ciphertext $c$, the decryption algorithm outputs the corresponding plaintext $m$.
\end{itemize}

We called that above public-key cryptosystem is additively homomorphic if it satisfies the following properties.

\begin{itemize}
  \item Given two ciphertexts $c_1 = \mathsf{Enc}_{pk}(m_1)$ and $c_2 = \mathsf{Enc}_{pk}(m_2)$, it holds that $\mathsf{Dec}_{sk}(c_1 \cdot c_2) = m_1 + m_2$.
  \item Given a constant $a$ and a ciphertext $c=\mathsf{Enc}_{pk}(m)$, it holds that $\mathsf{Dec}_{sk}(c^a) = a\cdot m$.
\end{itemize}


\iffalse
\subsection{ABY2.0}

% for secure two-party computation,
ABY2.0 is an efficient mixed-protocol framework which allows two parties to jointly evaluate a function based on their private inputs~\cite{patra_aby20_2020}.
In this paper, we make use of arithmetic sharing and multiplication protocol of ABY2.0 as the building blocks.
Specificly, we use $\{S_0, S_1\}$ to denote the two parties, and all protocols are executed over an $\ell$-bit ring denoted by $\mathbb{Z}_{2^\ell}$.
For a value $v\in\mathbb{Z}_{2^\ell}$, there are two sharing semantics in ABY2.0, as denoted by $[\cdot]$-sharing and $\langle \cdot \rangle$-sharing as follows.

\textit{$[\cdot]$-sharing:} A value $v\in\mathbb{Z}_{2^\ell}$ is said to be $[\cdot]$-sharing among $\{S_0, S_1\}$, which means the party $S_i$ for $i\in\{0,1\}$ holds $[v]_i$ such that $v = [v]_0 + [v]_1$.

\textit{$\langle \cdot \rangle$-sharing:} A value $v\in\mathbb{Z}_{2^\ell}$ is said to be $\langle \cdot \rangle$-sharing among $\{S_0, S_1\}$, which means the party $S_i$ for $i\in\{0,1\}$ holds $\langle v \rangle_i = (\Delta_v, [\delta_v]_i)$, where $\Delta_v = v + \delta_v$.

It is clear that both $[\cdot]$-sharing and $\langle \cdot \rangle$-sharing supoort linear operations.
For example, given the $\langle \cdot \rangle$-sharing of $a,b$ and public constants $c_1,c_2$, $S_i$ can locally compute $\langle y_i \rangle = c_1 \cdot \langle a \rangle_i + c_2 \cdot \langle b \rangle_i$.
But the multiplication protocol is non-trivial, we give a brief describtion at a high level.
For more detail, please refer to~\cite{patra_aby20_2020}.

The multiplication protocol enables two parties to generate $\langle y \rangle$ where $y = ab$ when given the $\langle \cdot \rangle$-sharing of $a,b$.
\fi

% Specifically, we 

\section{The Proposed Scheme}\label{sec5}

In this section, we describe the proposed scheme in detail.

\subsection{Overview}

% 设计目标场景
As we mentioned above, our privacy-preserving truth discovery scheme is designed for scenarios where the sensory data provided by workers are sparse.
In the proposed scheme, workers need to upload indication vectors to mark which objects they observed.
Therefore, it is a challenge to protect the privacy of indication vectors, sensory data and workers' weight at the same time.
Although applying GC is a straightforward method to achieve strong privacy preservation, the generation of GC is still a cumbersome task.
% 包含

To protect workers' privacy and make the procedure efficient, we design a novel protocol by adopting additively homomorphic cryptosystem.
The whole procedure can be divided into three phases: {\em Initialization Phase}, {\em Sensing Phase}, {\em Iteration Phase}.
% At a high level, 

\subsection{Initialization Phase}

As described in Section~\ref{sec3-A}, assume that there are $K$ participating workers and $M$ sensing objects in the system.
In this phase, $S_0$ first generates an asymmetric key pair $(pk, sk)$ of the additively homomorphic cryptosystem by invoking $\mathsf{KeyGen}(\cdot)$ in Section~\ref{sec4-b}.
Then $S_0$ sets a small number for the significant level $\alpha$ ($\alpha$ is set to 0.05 by default), and publishes the public key $pk$ and significant level $\alpha$.

% the public key $pk$ and private $sk$ by invoking $\mathsf{KeyGen}(\cdot)$ described in Section~\ref{sec4-b}.

\subsection{Report Phase}

In this phase, each worker collects sensory data for distinct objects.
Taking worker $k$ as an example, it can obtains a set of sensory data $\{x_m^k\}_{m=1}^M$ and an indication vector $\{\phi_m^k\}_{m=1}^M$.

% 有的 worker 不想做全部的任务 (放到 introduction 部分)
% After that, each worker obtains a set of sensory data and an indication vector


\subsection{Secure Weight Update Phase}
\subsection{Secure Truth Update Phase}


\section{Performance Evaluation}

\bibliographystyle{IEEEtran}
\bibliography{ref}

\vspace{12pt}

\end{document}