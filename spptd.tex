\documentclass[conference]{IEEEtran}
\IEEEoverridecommandlockouts
% The preceding line is only needed to identify funding in the first footnote. If that is unneeded, please comment it out.
\usepackage{cite}
\usepackage{amsmath,amssymb,amsfonts}
\usepackage{algorithmic}
\usepackage{graphicx}
\usepackage{textcomp}
\usepackage[OT1]{fontenc}
\usepackage{xcolor}
% \usepackage[colorlinks]{hyperref}
\def\BibTeX{{\rm B\kern-.05em{\sc i\kern-.025em b}\kern-.08em
    T\kern-.1667em\lower.7ex\hbox{E}\kern-.125emX}}
\begin{document}

\title{Privacy-preserving truth discovery for sparse data in crowd sensing systems\\
% {\footnotesize \textsuperscript{*}Note: Sub-titles are not captured in Xplore and
% should not be used}
% \thanks{Identify applicable funding agency here. If none, delete this.}
}

\author{\IEEEauthorblockN{xxx}
\IEEEauthorblockA{\textit{Department of Electronic and Engineering} \\
\textit{University of Science and Technology of China}\\
Hefei, Anhui 230027, China \\
xxx@mail.ustc.edu.cn}
}
\maketitle

\begin{abstract}
% 1.Motivation 2.Problem Statement 3.Approach 4.Results 5.Conclusions

\end{abstract}

\begin{IEEEkeywords}
Crowd Sensing, Truth Discovery, Privacy Preservation
\end{IEEEkeywords}

\section{Introduction}\label{sec1}
% What is the problem, why is the problem is important
Recently, developments in the field of cloud computing and Internet of Things (IoT) have led to an increased interest in crowd sensing.
With the help of crowd sensing, a crowd server can collect and analyze the sensory data from different devices.


\section{Related Work}

\section{Problem Statement}

\section{Preliminaries}

\section{The Proposed Scheme}

\section{System Analysis}

\bibliographystyle{ieeetran}
\bibliography{ref}

\vspace{12pt}

\end{document}